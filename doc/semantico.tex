%%%%%%%%%%%%%%%%%%%%%%%%%%%%%%%%%%%%%%%%%%%%%%%%%%%%%%%%%%%%%%%%%%%%%%
% How to use writeLaTeX: 
%
% You edit the source code here on the left, and the preview on the
% right shows you the result within a few seconds.
%
% Bookmark this page and share the URL with your co-authors. They can
% edit at the same time!
%
% You can upload figures, bibliographies, custom classes and
% styles using the files menu.
%
%%%%%%%%%%%%%%%%%%%%%%%%%%%%%%%%%%%%%%%%%%%%%%%%%%%%%%%%%%%%%%%%%%%%%%

\documentclass[12pt]{article}
\usepackage{float}
\usepackage{sbc-template}
\usepackage{graphicx,url}
\usepackage[brazil]{babel}   
\usepackage[utf8]{inputenc}  
\usepackage{indentfirst}
\usepackage{caption3}
\usepackage{color}
\usepackage{alltt}
\usepackage{listings}

\lstset{ %
language=C,                % choose the language of the code
basicstyle=\footnotesize,       % the size of the fonts that are used for the code
numbers=left,                   % where to put the line-numbers
numberstyle=\footnotesize,      % the size of the fonts that are used for the line-numbers
stepnumber=1,                   % the step between two line-numbers. If it is 1 each line will be numbered
numbersep=7pt,                  % how far the line-numbers are from the code
backgroundcolor=\color{white},  % choose the background color. You must add \usepackage{color}
showspaces=false,               % show spaces adding particular underscores
showstringspaces=false,         % underline spaces within strings
showtabs=false,                 % show tabs within strings adding particular underscores
frame=single,           % adds a frame around the code
tabsize=2,          % sets default tabsize to 2 spaces
breaklines=true,        % sets automatic line breaking
breakatwhitespace=false,    % sets if automatic breaks should only happen at whitespace
escapeinside={\%*}{*)},          % if you want to add a comment within your code
extendedchars=true,
emph={%  
    set, pair, bool%
    },emphstyle={\bfseries},
morekeywords={set, pair}
literate={á}{{\'a}}1 {ã}{{\~a}}1 {é}{{\'e}}1
}
     
\sloppy

\title{CSet - Analisador semântico}

\author{Daniella Angelos $^1$}

\address{Departamento de Ciência da Computação - Universidade de Brasília}

\begin{document} 

\maketitle
% %------------------------------------------------
% \begin{abstract}
%  This paper presents the problem of Multiple Sequence Alignment and how the A * algorithm obtains the optimal result of this problem. We also show some high-performance platforms that can be applied to this problem in order to improve the performance of the implementation and finally some results obtained.
% \end{abstract}

%------------------------------------------------
\section{Objetivo}
Este trabalho visa apresentar o analisador semântico desenvolvido para a linguagem CSet proposta anteriormente.

%------------------------------------------------
\section{Introdução}

Um compilador pode ser dividido em duas partes principais: análise e síntese. A parte de análise quebra o programa fonte em pedaços com significado e aplica uma estrutura gramatical a eles, informando possíveis erros que foram encontrados. Esta parte também é responsável por coletar informações sobre o programa fonte e armazená-las em uma estrutura de dados chamada \textit{tabela de símbolos}, que é passada ao longo das fases intermediárias da análise até a parte de síntese.

A análise semântica é a última fase da etapa de análise, utiliza a árvore sintática e a tabela de símbolos para checar a consistência semântica do código fonte, de acordo com a definição da linguagem.

\section{Gramática}

A gramática da linguagem apresentada anteriormente sofreu algumas alterações que estão destacadas a seguir.

\vspace{2mm}
\begin{alltt}
Outset          -> Function
                  \textbf{| Declaration}
                  | Outset Function
                  
Function       -> Type identifier ( ArgList ) CompoundStmt
                  
\textbf{ArgList        -> Arglistlist
                  | \(\varepsilon\)     

Arglistlist     -> Arg
                  | ArgList , Arg}
                  
Arg            -> Type identifier
                  
Declaration    -> Type IdentList
                  | Type Attr
                  
Type           -> int
                  | float
                  | char
                  | bool
                  | set << Type >>
                  | pair << Type , Type >>

\textbf{Identlist      -> Identlistlist
                  | \(\varepsilon\)
                  
IdentListlist  -> identifier , IdentList
                  | identifier}
                  
Stmt           -> WhileStmt
                  | Expr ;
                  | IfStmt
                  | CompoundStmt
                  | Declaration ;
                  | IO ; 
                  | ReturnStmt ;
                  
WhileStmt      -> while ( Expr ) Stmt
                  
IfStmt         -> if ( Expr ) CompoundStmt
                  | if ( Expr ) CompoundStmt ElsePart
                  
ElsePart       -> else Stmt
                  
IO             -> print ( str )
                  | print ( identifier )
                  | read ( identifier )
   
ReturnStmt     -> return Expr
                  | return
                  
CompoundStmt   -> \{ StmtList \}
                  
StmtList       -> StmtList Stmt
                  | \(\varepsilon\)
                  
Expr           -> Attr
                  | Rvalue
                  | FuncCall
                  
Attr           -> identifier = Expr
                  
Rvalue         -> Rvalue Compare LogicalOr
                  | LogicalOr
                               
Compare        -> ==
                  | <
                  | >
                  | <=
                  | >=
                  | !=
                  
LogicalOr      -> LogicalAnd || LogicalOr
                  | LogicalAnd
                  
LogicalAnd     -> Pertinence && LogicalAnd
                  | Pertinence
                  
Pertinence     -> Pertinence <?> Cartesian
                  | Cartesian
                  
Cartesian      -> Cartesian <*> Addition
                  | Addition
                  
Addition       -> Addition + Multiplication
                  | Addition - Multiplication
                  | Multiplication
                  
Multiplication -> Multiplication * Factor
                  | Multiplication / Factor
                  | Term
                  
Term           -> ( Expr )
                  | - Term
                  | + Term
                  | $ Term
                  | ! Term
                  | \{ FactorList \}
                  | ( Pair )
                  | Factor
                  
Factor         -> identifier
                  | boolean
                  | number
                  | character
                  
FactorList     -> FactorList , Factor
                  | Factor
                  | \(\varepsilon\)
                  
Pair           -> Factor , Factor

FuncCall       -> identifier ( IdentList )

\end{alltt}

\vspace{2mm}

A inserção da regra \texttt{Outset -> Declaration} teve como objetivo permitir a declaração de variáveis globais.

Uma pequena alteração nas regras de lista de argumentos e \textit{ids} também foi realizado, pois a versão antiga permitia que ambas possuísse um número arbitrário de vírgulas sem que houvesse elementos entre elas.

\section{Léxico}

Erros léxicos correspondem a caracteres inválidos, em CSet são, por exemplo 'ç', '@' etc, e nomes de identificadores começados por números. Para capturar este erro de identificadores, a seguinte regra foi criada.

\begin{verbatim}
{D}+(_|{L}|{D})*	{return -2;}
\end{verbatim}

Se um token não casar com nenhuma regra, será capturado pela regra a seguir.

\begin{verbatim}
.				{return -1;}
\end{verbatim}

O retorno de um número negativo ao token corresponde a um erro. Este será, então, mostrado ao usuário com informações de linha, coluna, conteúdo e tipo de erro. Neste caso, a função \texttt{tratar\_erros} é chamada, ela é responsável por gerar a mensagem correspondente.

% \vspace{1mm}
% \begin{lstlisting}
% set<int> nat = {1, 2, 3};
% pair<int, bool> bah = (2, false);
% bool pert = bah <?> nat <*> {true, false};
% \end{lstlisting}

\section{Sintático}

Erros sintáticos estão relacionados à estrutura incorreta do programa. Trata-se de tokens que foram validados pelo analisador léxico mas aparecem em um local não esperado. Por exemplo, lexicamente, podemos escrever

\texttt{int int;}

Mas não sintaticamente, pois \texttt{int} corresponde a uma palavra reservada de tipo primitivo da linguagem, e depois de sua ocorrência é esperado um token de identificador, ou seja, esta linha não irá corresponder a nenhuma regra da gramática da linguagem, ocasionando um erro sintático.

Para que o analisador sintático possa identificar o maior número de erros possíveis, uma regra adicional foi acrescentada a algumas variáveis para que pudessem lidar com erros. Por exemplo:
 	
\begin{verbatim}
| error '}' { synerrors++; yyerrok; } 
\end{verbatim}

Faz parte da variável \texttt{function} e toda vez que um erro é encontrado, o mesmo é reportado e o analisador ignora qualquer token que for enviado pelo léxico até encontrar \texttt{'\}'}.

% \vspace{1mm}
% \begin{lstlisting}
% set<int> nat = {1, 2, 3};
% pair<int, bool> bah = (2, false);
% bool pert = bah <?> nat <*> {true, false};
% \end{lstlisting}

\subsection{Árvore sintática}

Para cada uma das variáveis da gramática (com uma ou outra exceção, que puderam utilizar uma definição mais genérica) foi criada uma estrutura que define o tipo do nó que é gerado após a regra ser atingida.

Cada uma dessas estruturas possuem campos referentes à variável e correspondem, em sua maioria, a outros nós na árvore.

\subsection{Tabela de símbolos}

Para armazenar informações sobre identificadores de variáveis e funções, uma tabela de símbolos é criada e que será de extrema utilidade ao analisador semântico.

Ao encontrar um identificador, o sintático verifica se a entrada já existe na tabela, caso não exista, a mesma é criada.

\section{Semântico}

Com o auxílio da tabela de símbolos e da árvore sintática, o analisador semãntico encontra inconsistências no arquivo fonte que dizem respeito, principalmente, à tipagem das expressões e à declaração de variáveis.

Os erros que este analisador captura estão listados a seguir:
\begin{itemize}
\item Inconsistência de tipos em uma atribuição de variável
\item Variáveis usadas, mas não declaradas
\item Chamada a uma função não declarada
\item Chamada a uma função com o número inválido de argumentos
\item Passagem de argumentos a uma função com tipos que divergem dos parâmetros da mesma
\item Condição em expressões \texttt{if} e \texttt{while} com tipo diferente de \texttt{bool}
\end{itemize}

Para as verificações de tipo, foi implementado uma checagem de tipo que analisa a tipagem de cada um dos termos que compõem a expressão.

\section{Geração de código}

A geração de código intermediária foi iniciada, apesar de não finalizada. Está sendo utilizado o formato TAC.

Chamadas a função geram o código e um esqueleto é produzido para expressões \texttt{if} e \texttt{while}. Também já cria-se os rótulos para procedimentos.

O código que está sendo gerado é salvo no arquivo cset.tac ao final da execução do compilador.

\section{Exemplos}

Este documento foi entregue em conjunto com 4 exemplos de programas em CSet. Dentre eles, 2 contendo códigos lexicamente, sintaticamente e semanticamente corretos, e 2 com erros semânticos.

\begin{itemize}
\item Corretos: \textit{example(1,2).cset}
\item Com erros: \textit{error\_example(1,2).cset}
\end{itemize}

Erros:
\begin{itemize}
\item Arquivo 1:
\begin{itemize}
\item Erro semantico linha 4. Tipos incompativeis na atribuicao.
\item Erro semantivo linha 8. Condicao em while deve ter tipo bool.
\item Erro semantico, linha 17. Tipo de argumento nao esperado.
\end{itemize}

\item Arquivo 2:
\begin{itemize}
\item Erro semantico, linha 6. Chamada a funcao nao definida!
\item Erro semantico linha 7. O tipo do lado esquerdo da atribuicao nao pode ser avaliado.
\item Erro semantico. Identificador `c' nao declarado.
\end{itemize}
\end{itemize}

\section{Considerações finais}

Para compilar o programa basta rodar \texttt{make} no diretório que o contém.
Para executar: \texttt{./cset <input.file>}

Algumas dificuldades foram enfrentadas que, inclusive, ocuparam muito tempo do desenvolvimento. 
Uma das maiores foi a realização da checagem de tipo.

\begin{thebibliography}{1}
\bibitem{book}
A.~V.~Abo, M.~S.~Lam, R.~Sethi, J.~D.~Ullman, \emph{Compilers - Principles, Techniques and Tools}
\hskip 1em plus
	0.5em minus 0.4em\relax 2nd ed. 1986

\end{thebibliography}

%------------------------------------------------

\end{document}
